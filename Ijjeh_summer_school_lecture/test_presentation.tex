\documentclass{beamer}

\usetheme{Madrid}
\usecolortheme{default}

\title{Artificial Intelligence Approaches for Structural Health Monitoring}
\author{Your Name}
\date{\today}

\begin{document}
	
	\begin{frame}
		\titlepage
	\end{frame}
	
	\begin{frame}{Introduction}
		\frametitle{Structural Health Monitoring (SHM)}
		\begin{itemize}
			\item Structural Health Monitoring (SHM) refers to the process of continuously monitoring and evaluating the condition of structures to ensure their safety and performance.
			\item Traditional SHM techniques involve manual inspections and periodic measurements, which can be time-consuming and costly.
			\item Artificial Intelligence (AI) approaches offer innovative solutions to enhance SHM by automating data analysis, enabling real-time monitoring, and detecting structural abnormalities.
		\end{itemize}
	\end{frame}
	
	\begin{frame}{AI Approaches for SHM}
		\begin{itemize}
			\item Machine Learning-based SHM:
			\begin{itemize}
				\item Utilizing machine learning algorithms, such as Support Vector Machines (SVM), Random Forests, or Neural Networks, to analyze sensor data and identify structural anomalies.
				\item Training models using labeled data from healthy and damaged structures to enable accurate classification or regression.
				\item Enabling predictive maintenance by detecting early signs of structural degradation.
			\end{itemize}
			\item Deep Learning-based SHM:
			\begin{itemize}
				\item Leveraging deep neural networks, such as Convolutional Neural Networks (CNN) or Recurrent Neural Networks (RNN), to analyze sensor data for automatic feature extraction and damage detection.
				\item Handling large-scale data, such as images or time-series signals, to capture complex patterns and correlations in structural behavior.
			\end{itemize}
		\end{itemize}
	\end{frame}
	
	\begin{frame}{AI Approaches for SHM (contd.)}
		\begin{itemize}
			\item Reinforcement Learning-based SHM:
			\begin{itemize}
				\item Applying reinforcement learning techniques to develop intelligent agents that interact with the structure and learn optimal control strategies for maintenance and structural health preservation.
				\item Training agents to make decisions on inspection schedules, sensor placement, or structural maintenance actions.
			\end{itemize}
			\item Hybrid Approaches:
			\begin{itemize}
				\item Integrating multiple AI techniques, such as combining machine learning and physics-based models, to improve the accuracy and interpretability of SHM systems.
				\item Incorporating domain knowledge and expert systems to enhance AI-based approaches with structural engineering expertise.
			\end{itemize}
		\end{itemize}
	\end{frame}
	
	\begin{frame}{Advantages and Challenges}
		\begin{itemize}
			\item Advantages of AI Approaches for SHM:
			\begin{itemize}
				\item Continuous monitoring and real-time detection of structural anomalies.
				\item Improved accuracy and efficiency compared to traditional manual inspections.
				\item Early detection of damage, enabling proactive maintenance and avoiding catastrophic failures.
			\end{itemize}
			\item Challenges and Considerations:
			\begin{itemize}
				\item Data availability and quality for training AI models.
				\item Interpret-ability and explain-ability of AI-based SHM systems.
				\item Generalization and transferability of trained models to new structures or conditions.
				\item Integration of AI with existing SHM
			\end{itemize}
		\end{itemize}
		\end{frame}
\end{document}