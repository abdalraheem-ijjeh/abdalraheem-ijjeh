\documentclass[10pt,aspectratio=169,dvipsnames]{beamer}


%\usepackage{appendixnumberbeamer}
%\usepackage{booktabs}
%\usepackage{csvsimple} % for csv read
%\usepackage[scale=2]{ccicons}
%\usepackage{pgfplots}

%\usepackage{amsmath, nccmath} % fleqn
\usepackage{amssymb}
\usepackage{xspace}
\usepackage{xcolor}

\usepackage{totcount}
\usepackage{tikz}
\usepackage{bm}
\usepackage{float}
\usepackage{eso-pic} 
\usepackage{wrapfig}
\usepackage{animate,media9,movie15}
\usepackage{subfig}
\usepackage{fancybox}
%\usepackage{multimedia}
\usepackage{dashbox}
\usepackage{tcolorbox}
\usepackage{multicol}
\usepackage{tikz}

\usepackage{subfig}

\usepackage{xcolor}

\begin{document}
	\begin{frame}
		\noindent
		\alert{Convolution operation (cross-correlation)} \\ 
		known as a \alert{sliding dot product} or \alert{sliding inner-product}
		\begin{equation*}
			G[i,j]= k \otimes F  = {\sum_{a}^{}\sum_{b}k(a,b)F(i+a, j+b)}
		\end{equation*}	
		
		\begin{equation*}
			h_n= \frac{h+2 \times p -h_k}{s} + 1\hspace{20} \& \hspace{20} w_n= \frac{w+2 \times p -w_k}{s} + 1
		\end{equation*}	
	{\alert{\(h_n\)} and \alert{\(w_n\)}} are the new height and width dimensions of the feature map. \\
	\alert{\(p\)} is the padding added to the input image. \\
	{\alert{\(h_k\)} and \alert{\(w_k\)}} are the height and the width of the kernel. \\
	\alert{\(s\)} is the stride that defines how much the kernel slides each step during convolution.	
	\end{frame}	
\end{document}